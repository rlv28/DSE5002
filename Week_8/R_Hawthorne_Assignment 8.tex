\documentclass[11pt]{article}

    \usepackage[breakable]{tcolorbox}
    \usepackage{parskip} % Stop auto-indenting (to mimic markdown behaviour)
    

    % Basic figure setup, for now with no caption control since it's done
    % automatically by Pandoc (which extracts ![](path) syntax from Markdown).
    \usepackage{graphicx}
    % Maintain compatibility with old templates. Remove in nbconvert 6.0
    \let\Oldincludegraphics\includegraphics
    % Ensure that by default, figures have no caption (until we provide a
    % proper Figure object with a Caption API and a way to capture that
    % in the conversion process - todo).
    \usepackage{caption}
    \DeclareCaptionFormat{nocaption}{}
    \captionsetup{format=nocaption,aboveskip=0pt,belowskip=0pt}

    \usepackage{float}
    \floatplacement{figure}{H} % forces figures to be placed at the correct location
    \usepackage{xcolor} % Allow colors to be defined
    \usepackage{enumerate} % Needed for markdown enumerations to work
    \usepackage{geometry} % Used to adjust the document margins
    \usepackage{amsmath} % Equations
    \usepackage{amssymb} % Equations
    \usepackage{textcomp} % defines textquotesingle
    % Hack from http://tex.stackexchange.com/a/47451/13684:
    \AtBeginDocument{%
        \def\PYZsq{\textquotesingle}% Upright quotes in Pygmentized code
    }
    \usepackage{upquote} % Upright quotes for verbatim code
    \usepackage{eurosym} % defines \euro

    \usepackage{iftex}
    \ifPDFTeX
        \usepackage[T1]{fontenc}
        \IfFileExists{alphabeta.sty}{
              \usepackage{alphabeta}
          }{
              \usepackage[mathletters]{ucs}
              \usepackage[utf8x]{inputenc}
          }
    \else
        \usepackage{fontspec}
        \usepackage{unicode-math}
    \fi

    \usepackage{fancyvrb} % verbatim replacement that allows latex
    \usepackage{grffile} % extends the file name processing of package graphics
                         % to support a larger range
    \makeatletter % fix for old versions of grffile with XeLaTeX
    \@ifpackagelater{grffile}{2019/11/01}
    {
      % Do nothing on new versions
    }
    {
      \def\Gread@@xetex#1{%
        \IfFileExists{"\Gin@base".bb}%
        {\Gread@eps{\Gin@base.bb}}%
        {\Gread@@xetex@aux#1}%
      }
    }
    \makeatother
    \usepackage[Export]{adjustbox} % Used to constrain images to a maximum size
    \adjustboxset{max size={0.9\linewidth}{0.9\paperheight}}

    % The hyperref package gives us a pdf with properly built
    % internal navigation ('pdf bookmarks' for the table of contents,
    % internal cross-reference links, web links for URLs, etc.)
    \usepackage{hyperref}
    % The default LaTeX title has an obnoxious amount of whitespace. By default,
    % titling removes some of it. It also provides customization options.
    \usepackage{titling}
    \usepackage{longtable} % longtable support required by pandoc >1.10
    \usepackage{booktabs}  % table support for pandoc > 1.12.2
    \usepackage{array}     % table support for pandoc >= 2.11.3
    \usepackage{calc}      % table minipage width calculation for pandoc >= 2.11.1
    \usepackage[inline]{enumitem} % IRkernel/repr support (it uses the enumerate* environment)
    \usepackage[normalem]{ulem} % ulem is needed to support strikethroughs (\sout)
                                % normalem makes italics be italics, not underlines
    \usepackage{mathrsfs}
    

    
    % Colors for the hyperref package
    \definecolor{urlcolor}{rgb}{0,.145,.698}
    \definecolor{linkcolor}{rgb}{.71,0.21,0.01}
    \definecolor{citecolor}{rgb}{.12,.54,.11}

    % ANSI colors
    \definecolor{ansi-black}{HTML}{3E424D}
    \definecolor{ansi-black-intense}{HTML}{282C36}
    \definecolor{ansi-red}{HTML}{E75C58}
    \definecolor{ansi-red-intense}{HTML}{B22B31}
    \definecolor{ansi-green}{HTML}{00A250}
    \definecolor{ansi-green-intense}{HTML}{007427}
    \definecolor{ansi-yellow}{HTML}{DDB62B}
    \definecolor{ansi-yellow-intense}{HTML}{B27D12}
    \definecolor{ansi-blue}{HTML}{208FFB}
    \definecolor{ansi-blue-intense}{HTML}{0065CA}
    \definecolor{ansi-magenta}{HTML}{D160C4}
    \definecolor{ansi-magenta-intense}{HTML}{A03196}
    \definecolor{ansi-cyan}{HTML}{60C6C8}
    \definecolor{ansi-cyan-intense}{HTML}{258F8F}
    \definecolor{ansi-white}{HTML}{C5C1B4}
    \definecolor{ansi-white-intense}{HTML}{A1A6B2}
    \definecolor{ansi-default-inverse-fg}{HTML}{FFFFFF}
    \definecolor{ansi-default-inverse-bg}{HTML}{000000}

    % common color for the border for error outputs.
    \definecolor{outerrorbackground}{HTML}{FFDFDF}

    % commands and environments needed by pandoc snippets
    % extracted from the output of `pandoc -s`
    \providecommand{\tightlist}{%
      \setlength{\itemsep}{0pt}\setlength{\parskip}{0pt}}
    \DefineVerbatimEnvironment{Highlighting}{Verbatim}{commandchars=\\\{\}}
    % Add ',fontsize=\small' for more characters per line
    \newenvironment{Shaded}{}{}
    \newcommand{\KeywordTok}[1]{\textcolor[rgb]{0.00,0.44,0.13}{\textbf{{#1}}}}
    \newcommand{\DataTypeTok}[1]{\textcolor[rgb]{0.56,0.13,0.00}{{#1}}}
    \newcommand{\DecValTok}[1]{\textcolor[rgb]{0.25,0.63,0.44}{{#1}}}
    \newcommand{\BaseNTok}[1]{\textcolor[rgb]{0.25,0.63,0.44}{{#1}}}
    \newcommand{\FloatTok}[1]{\textcolor[rgb]{0.25,0.63,0.44}{{#1}}}
    \newcommand{\CharTok}[1]{\textcolor[rgb]{0.25,0.44,0.63}{{#1}}}
    \newcommand{\StringTok}[1]{\textcolor[rgb]{0.25,0.44,0.63}{{#1}}}
    \newcommand{\CommentTok}[1]{\textcolor[rgb]{0.38,0.63,0.69}{\textit{{#1}}}}
    \newcommand{\OtherTok}[1]{\textcolor[rgb]{0.00,0.44,0.13}{{#1}}}
    \newcommand{\AlertTok}[1]{\textcolor[rgb]{1.00,0.00,0.00}{\textbf{{#1}}}}
    \newcommand{\FunctionTok}[1]{\textcolor[rgb]{0.02,0.16,0.49}{{#1}}}
    \newcommand{\RegionMarkerTok}[1]{{#1}}
    \newcommand{\ErrorTok}[1]{\textcolor[rgb]{1.00,0.00,0.00}{\textbf{{#1}}}}
    \newcommand{\NormalTok}[1]{{#1}}

    % Additional commands for more recent versions of Pandoc
    \newcommand{\ConstantTok}[1]{\textcolor[rgb]{0.53,0.00,0.00}{{#1}}}
    \newcommand{\SpecialCharTok}[1]{\textcolor[rgb]{0.25,0.44,0.63}{{#1}}}
    \newcommand{\VerbatimStringTok}[1]{\textcolor[rgb]{0.25,0.44,0.63}{{#1}}}
    \newcommand{\SpecialStringTok}[1]{\textcolor[rgb]{0.73,0.40,0.53}{{#1}}}
    \newcommand{\ImportTok}[1]{{#1}}
    \newcommand{\DocumentationTok}[1]{\textcolor[rgb]{0.73,0.13,0.13}{\textit{{#1}}}}
    \newcommand{\AnnotationTok}[1]{\textcolor[rgb]{0.38,0.63,0.69}{\textbf{\textit{{#1}}}}}
    \newcommand{\CommentVarTok}[1]{\textcolor[rgb]{0.38,0.63,0.69}{\textbf{\textit{{#1}}}}}
    \newcommand{\VariableTok}[1]{\textcolor[rgb]{0.10,0.09,0.49}{{#1}}}
    \newcommand{\ControlFlowTok}[1]{\textcolor[rgb]{0.00,0.44,0.13}{\textbf{{#1}}}}
    \newcommand{\OperatorTok}[1]{\textcolor[rgb]{0.40,0.40,0.40}{{#1}}}
    \newcommand{\BuiltInTok}[1]{{#1}}
    \newcommand{\ExtensionTok}[1]{{#1}}
    \newcommand{\PreprocessorTok}[1]{\textcolor[rgb]{0.74,0.48,0.00}{{#1}}}
    \newcommand{\AttributeTok}[1]{\textcolor[rgb]{0.49,0.56,0.16}{{#1}}}
    \newcommand{\InformationTok}[1]{\textcolor[rgb]{0.38,0.63,0.69}{\textbf{\textit{{#1}}}}}
    \newcommand{\WarningTok}[1]{\textcolor[rgb]{0.38,0.63,0.69}{\textbf{\textit{{#1}}}}}


    % Define a nice break command that doesn't care if a line doesn't already
    % exist.
    \def\br{\hspace*{\fill} \\* }
    % Math Jax compatibility definitions
    \def\gt{>}
    \def\lt{<}
    \let\Oldtex\TeX
    \let\Oldlatex\LaTeX
    \renewcommand{\TeX}{\textrm{\Oldtex}}
    \renewcommand{\LaTeX}{\textrm{\Oldlatex}}
    % Document parameters
    % Document title
    \title{R\_Hawthorne\_Assignment 8}
    
    
    
    
    
% Pygments definitions
\makeatletter
\def\PY@reset{\let\PY@it=\relax \let\PY@bf=\relax%
    \let\PY@ul=\relax \let\PY@tc=\relax%
    \let\PY@bc=\relax \let\PY@ff=\relax}
\def\PY@tok#1{\csname PY@tok@#1\endcsname}
\def\PY@toks#1+{\ifx\relax#1\empty\else%
    \PY@tok{#1}\expandafter\PY@toks\fi}
\def\PY@do#1{\PY@bc{\PY@tc{\PY@ul{%
    \PY@it{\PY@bf{\PY@ff{#1}}}}}}}
\def\PY#1#2{\PY@reset\PY@toks#1+\relax+\PY@do{#2}}

\@namedef{PY@tok@w}{\def\PY@tc##1{\textcolor[rgb]{0.73,0.73,0.73}{##1}}}
\@namedef{PY@tok@c}{\let\PY@it=\textit\def\PY@tc##1{\textcolor[rgb]{0.24,0.48,0.48}{##1}}}
\@namedef{PY@tok@cp}{\def\PY@tc##1{\textcolor[rgb]{0.61,0.40,0.00}{##1}}}
\@namedef{PY@tok@k}{\let\PY@bf=\textbf\def\PY@tc##1{\textcolor[rgb]{0.00,0.50,0.00}{##1}}}
\@namedef{PY@tok@kp}{\def\PY@tc##1{\textcolor[rgb]{0.00,0.50,0.00}{##1}}}
\@namedef{PY@tok@kt}{\def\PY@tc##1{\textcolor[rgb]{0.69,0.00,0.25}{##1}}}
\@namedef{PY@tok@o}{\def\PY@tc##1{\textcolor[rgb]{0.40,0.40,0.40}{##1}}}
\@namedef{PY@tok@ow}{\let\PY@bf=\textbf\def\PY@tc##1{\textcolor[rgb]{0.67,0.13,1.00}{##1}}}
\@namedef{PY@tok@nb}{\def\PY@tc##1{\textcolor[rgb]{0.00,0.50,0.00}{##1}}}
\@namedef{PY@tok@nf}{\def\PY@tc##1{\textcolor[rgb]{0.00,0.00,1.00}{##1}}}
\@namedef{PY@tok@nc}{\let\PY@bf=\textbf\def\PY@tc##1{\textcolor[rgb]{0.00,0.00,1.00}{##1}}}
\@namedef{PY@tok@nn}{\let\PY@bf=\textbf\def\PY@tc##1{\textcolor[rgb]{0.00,0.00,1.00}{##1}}}
\@namedef{PY@tok@ne}{\let\PY@bf=\textbf\def\PY@tc##1{\textcolor[rgb]{0.80,0.25,0.22}{##1}}}
\@namedef{PY@tok@nv}{\def\PY@tc##1{\textcolor[rgb]{0.10,0.09,0.49}{##1}}}
\@namedef{PY@tok@no}{\def\PY@tc##1{\textcolor[rgb]{0.53,0.00,0.00}{##1}}}
\@namedef{PY@tok@nl}{\def\PY@tc##1{\textcolor[rgb]{0.46,0.46,0.00}{##1}}}
\@namedef{PY@tok@ni}{\let\PY@bf=\textbf\def\PY@tc##1{\textcolor[rgb]{0.44,0.44,0.44}{##1}}}
\@namedef{PY@tok@na}{\def\PY@tc##1{\textcolor[rgb]{0.41,0.47,0.13}{##1}}}
\@namedef{PY@tok@nt}{\let\PY@bf=\textbf\def\PY@tc##1{\textcolor[rgb]{0.00,0.50,0.00}{##1}}}
\@namedef{PY@tok@nd}{\def\PY@tc##1{\textcolor[rgb]{0.67,0.13,1.00}{##1}}}
\@namedef{PY@tok@s}{\def\PY@tc##1{\textcolor[rgb]{0.73,0.13,0.13}{##1}}}
\@namedef{PY@tok@sd}{\let\PY@it=\textit\def\PY@tc##1{\textcolor[rgb]{0.73,0.13,0.13}{##1}}}
\@namedef{PY@tok@si}{\let\PY@bf=\textbf\def\PY@tc##1{\textcolor[rgb]{0.64,0.35,0.47}{##1}}}
\@namedef{PY@tok@se}{\let\PY@bf=\textbf\def\PY@tc##1{\textcolor[rgb]{0.67,0.36,0.12}{##1}}}
\@namedef{PY@tok@sr}{\def\PY@tc##1{\textcolor[rgb]{0.64,0.35,0.47}{##1}}}
\@namedef{PY@tok@ss}{\def\PY@tc##1{\textcolor[rgb]{0.10,0.09,0.49}{##1}}}
\@namedef{PY@tok@sx}{\def\PY@tc##1{\textcolor[rgb]{0.00,0.50,0.00}{##1}}}
\@namedef{PY@tok@m}{\def\PY@tc##1{\textcolor[rgb]{0.40,0.40,0.40}{##1}}}
\@namedef{PY@tok@gh}{\let\PY@bf=\textbf\def\PY@tc##1{\textcolor[rgb]{0.00,0.00,0.50}{##1}}}
\@namedef{PY@tok@gu}{\let\PY@bf=\textbf\def\PY@tc##1{\textcolor[rgb]{0.50,0.00,0.50}{##1}}}
\@namedef{PY@tok@gd}{\def\PY@tc##1{\textcolor[rgb]{0.63,0.00,0.00}{##1}}}
\@namedef{PY@tok@gi}{\def\PY@tc##1{\textcolor[rgb]{0.00,0.52,0.00}{##1}}}
\@namedef{PY@tok@gr}{\def\PY@tc##1{\textcolor[rgb]{0.89,0.00,0.00}{##1}}}
\@namedef{PY@tok@ge}{\let\PY@it=\textit}
\@namedef{PY@tok@gs}{\let\PY@bf=\textbf}
\@namedef{PY@tok@gp}{\let\PY@bf=\textbf\def\PY@tc##1{\textcolor[rgb]{0.00,0.00,0.50}{##1}}}
\@namedef{PY@tok@go}{\def\PY@tc##1{\textcolor[rgb]{0.44,0.44,0.44}{##1}}}
\@namedef{PY@tok@gt}{\def\PY@tc##1{\textcolor[rgb]{0.00,0.27,0.87}{##1}}}
\@namedef{PY@tok@err}{\def\PY@bc##1{{\setlength{\fboxsep}{\string -\fboxrule}\fcolorbox[rgb]{1.00,0.00,0.00}{1,1,1}{\strut ##1}}}}
\@namedef{PY@tok@kc}{\let\PY@bf=\textbf\def\PY@tc##1{\textcolor[rgb]{0.00,0.50,0.00}{##1}}}
\@namedef{PY@tok@kd}{\let\PY@bf=\textbf\def\PY@tc##1{\textcolor[rgb]{0.00,0.50,0.00}{##1}}}
\@namedef{PY@tok@kn}{\let\PY@bf=\textbf\def\PY@tc##1{\textcolor[rgb]{0.00,0.50,0.00}{##1}}}
\@namedef{PY@tok@kr}{\let\PY@bf=\textbf\def\PY@tc##1{\textcolor[rgb]{0.00,0.50,0.00}{##1}}}
\@namedef{PY@tok@bp}{\def\PY@tc##1{\textcolor[rgb]{0.00,0.50,0.00}{##1}}}
\@namedef{PY@tok@fm}{\def\PY@tc##1{\textcolor[rgb]{0.00,0.00,1.00}{##1}}}
\@namedef{PY@tok@vc}{\def\PY@tc##1{\textcolor[rgb]{0.10,0.09,0.49}{##1}}}
\@namedef{PY@tok@vg}{\def\PY@tc##1{\textcolor[rgb]{0.10,0.09,0.49}{##1}}}
\@namedef{PY@tok@vi}{\def\PY@tc##1{\textcolor[rgb]{0.10,0.09,0.49}{##1}}}
\@namedef{PY@tok@vm}{\def\PY@tc##1{\textcolor[rgb]{0.10,0.09,0.49}{##1}}}
\@namedef{PY@tok@sa}{\def\PY@tc##1{\textcolor[rgb]{0.73,0.13,0.13}{##1}}}
\@namedef{PY@tok@sb}{\def\PY@tc##1{\textcolor[rgb]{0.73,0.13,0.13}{##1}}}
\@namedef{PY@tok@sc}{\def\PY@tc##1{\textcolor[rgb]{0.73,0.13,0.13}{##1}}}
\@namedef{PY@tok@dl}{\def\PY@tc##1{\textcolor[rgb]{0.73,0.13,0.13}{##1}}}
\@namedef{PY@tok@s2}{\def\PY@tc##1{\textcolor[rgb]{0.73,0.13,0.13}{##1}}}
\@namedef{PY@tok@sh}{\def\PY@tc##1{\textcolor[rgb]{0.73,0.13,0.13}{##1}}}
\@namedef{PY@tok@s1}{\def\PY@tc##1{\textcolor[rgb]{0.73,0.13,0.13}{##1}}}
\@namedef{PY@tok@mb}{\def\PY@tc##1{\textcolor[rgb]{0.40,0.40,0.40}{##1}}}
\@namedef{PY@tok@mf}{\def\PY@tc##1{\textcolor[rgb]{0.40,0.40,0.40}{##1}}}
\@namedef{PY@tok@mh}{\def\PY@tc##1{\textcolor[rgb]{0.40,0.40,0.40}{##1}}}
\@namedef{PY@tok@mi}{\def\PY@tc##1{\textcolor[rgb]{0.40,0.40,0.40}{##1}}}
\@namedef{PY@tok@il}{\def\PY@tc##1{\textcolor[rgb]{0.40,0.40,0.40}{##1}}}
\@namedef{PY@tok@mo}{\def\PY@tc##1{\textcolor[rgb]{0.40,0.40,0.40}{##1}}}
\@namedef{PY@tok@ch}{\let\PY@it=\textit\def\PY@tc##1{\textcolor[rgb]{0.24,0.48,0.48}{##1}}}
\@namedef{PY@tok@cm}{\let\PY@it=\textit\def\PY@tc##1{\textcolor[rgb]{0.24,0.48,0.48}{##1}}}
\@namedef{PY@tok@cpf}{\let\PY@it=\textit\def\PY@tc##1{\textcolor[rgb]{0.24,0.48,0.48}{##1}}}
\@namedef{PY@tok@c1}{\let\PY@it=\textit\def\PY@tc##1{\textcolor[rgb]{0.24,0.48,0.48}{##1}}}
\@namedef{PY@tok@cs}{\let\PY@it=\textit\def\PY@tc##1{\textcolor[rgb]{0.24,0.48,0.48}{##1}}}

\def\PYZbs{\char`\\}
\def\PYZus{\char`\_}
\def\PYZob{\char`\{}
\def\PYZcb{\char`\}}
\def\PYZca{\char`\^}
\def\PYZam{\char`\&}
\def\PYZlt{\char`\<}
\def\PYZgt{\char`\>}
\def\PYZsh{\char`\#}
\def\PYZpc{\char`\%}
\def\PYZdl{\char`\$}
\def\PYZhy{\char`\-}
\def\PYZsq{\char`\'}
\def\PYZdq{\char`\"}
\def\PYZti{\char`\~}
% for compatibility with earlier versions
\def\PYZat{@}
\def\PYZlb{[}
\def\PYZrb{]}
\makeatother


    % For linebreaks inside Verbatim environment from package fancyvrb.
    \makeatletter
        \newbox\Wrappedcontinuationbox
        \newbox\Wrappedvisiblespacebox
        \newcommand*\Wrappedvisiblespace {\textcolor{red}{\textvisiblespace}}
        \newcommand*\Wrappedcontinuationsymbol {\textcolor{red}{\llap{\tiny$\m@th\hookrightarrow$}}}
        \newcommand*\Wrappedcontinuationindent {3ex }
        \newcommand*\Wrappedafterbreak {\kern\Wrappedcontinuationindent\copy\Wrappedcontinuationbox}
        % Take advantage of the already applied Pygments mark-up to insert
        % potential linebreaks for TeX processing.
        %        {, <, #, %, $, ' and ": go to next line.
        %        _, }, ^, &, >, - and ~: stay at end of broken line.
        % Use of \textquotesingle for straight quote.
        \newcommand*\Wrappedbreaksatspecials {%
            \def\PYGZus{\discretionary{\char`\_}{\Wrappedafterbreak}{\char`\_}}%
            \def\PYGZob{\discretionary{}{\Wrappedafterbreak\char`\{}{\char`\{}}%
            \def\PYGZcb{\discretionary{\char`\}}{\Wrappedafterbreak}{\char`\}}}%
            \def\PYGZca{\discretionary{\char`\^}{\Wrappedafterbreak}{\char`\^}}%
            \def\PYGZam{\discretionary{\char`\&}{\Wrappedafterbreak}{\char`\&}}%
            \def\PYGZlt{\discretionary{}{\Wrappedafterbreak\char`\<}{\char`\<}}%
            \def\PYGZgt{\discretionary{\char`\>}{\Wrappedafterbreak}{\char`\>}}%
            \def\PYGZsh{\discretionary{}{\Wrappedafterbreak\char`\#}{\char`\#}}%
            \def\PYGZpc{\discretionary{}{\Wrappedafterbreak\char`\%}{\char`\%}}%
            \def\PYGZdl{\discretionary{}{\Wrappedafterbreak\char`\$}{\char`\$}}%
            \def\PYGZhy{\discretionary{\char`\-}{\Wrappedafterbreak}{\char`\-}}%
            \def\PYGZsq{\discretionary{}{\Wrappedafterbreak\textquotesingle}{\textquotesingle}}%
            \def\PYGZdq{\discretionary{}{\Wrappedafterbreak\char`\"}{\char`\"}}%
            \def\PYGZti{\discretionary{\char`\~}{\Wrappedafterbreak}{\char`\~}}%
        }
        % Some characters . , ; ? ! / are not pygmentized.
        % This macro makes them "active" and they will insert potential linebreaks
        \newcommand*\Wrappedbreaksatpunct {%
            \lccode`\~`\.\lowercase{\def~}{\discretionary{\hbox{\char`\.}}{\Wrappedafterbreak}{\hbox{\char`\.}}}%
            \lccode`\~`\,\lowercase{\def~}{\discretionary{\hbox{\char`\,}}{\Wrappedafterbreak}{\hbox{\char`\,}}}%
            \lccode`\~`\;\lowercase{\def~}{\discretionary{\hbox{\char`\;}}{\Wrappedafterbreak}{\hbox{\char`\;}}}%
            \lccode`\~`\:\lowercase{\def~}{\discretionary{\hbox{\char`\:}}{\Wrappedafterbreak}{\hbox{\char`\:}}}%
            \lccode`\~`\?\lowercase{\def~}{\discretionary{\hbox{\char`\?}}{\Wrappedafterbreak}{\hbox{\char`\?}}}%
            \lccode`\~`\!\lowercase{\def~}{\discretionary{\hbox{\char`\!}}{\Wrappedafterbreak}{\hbox{\char`\!}}}%
            \lccode`\~`\/\lowercase{\def~}{\discretionary{\hbox{\char`\/}}{\Wrappedafterbreak}{\hbox{\char`\/}}}%
            \catcode`\.\active
            \catcode`\,\active
            \catcode`\;\active
            \catcode`\:\active
            \catcode`\?\active
            \catcode`\!\active
            \catcode`\/\active
            \lccode`\~`\~
        }
    \makeatother

    \let\OriginalVerbatim=\Verbatim
    \makeatletter
    \renewcommand{\Verbatim}[1][1]{%
        %\parskip\z@skip
        \sbox\Wrappedcontinuationbox {\Wrappedcontinuationsymbol}%
        \sbox\Wrappedvisiblespacebox {\FV@SetupFont\Wrappedvisiblespace}%
        \def\FancyVerbFormatLine ##1{\hsize\linewidth
            \vtop{\raggedright\hyphenpenalty\z@\exhyphenpenalty\z@
                \doublehyphendemerits\z@\finalhyphendemerits\z@
                \strut ##1\strut}%
        }%
        % If the linebreak is at a space, the latter will be displayed as visible
        % space at end of first line, and a continuation symbol starts next line.
        % Stretch/shrink are however usually zero for typewriter font.
        \def\FV@Space {%
            \nobreak\hskip\z@ plus\fontdimen3\font minus\fontdimen4\font
            \discretionary{\copy\Wrappedvisiblespacebox}{\Wrappedafterbreak}
            {\kern\fontdimen2\font}%
        }%

        % Allow breaks at special characters using \PYG... macros.
        \Wrappedbreaksatspecials
        % Breaks at punctuation characters . , ; ? ! and / need catcode=\active
        \OriginalVerbatim[#1,codes*=\Wrappedbreaksatpunct]%
    }
    \makeatother

    % Exact colors from NB
    \definecolor{incolor}{HTML}{303F9F}
    \definecolor{outcolor}{HTML}{D84315}
    \definecolor{cellborder}{HTML}{CFCFCF}
    \definecolor{cellbackground}{HTML}{F7F7F7}

    % prompt
    \makeatletter
    \newcommand{\boxspacing}{\kern\kvtcb@left@rule\kern\kvtcb@boxsep}
    \makeatother
    \newcommand{\prompt}[4]{
        {\ttfamily\llap{{\color{#2}[#3]:\hspace{3pt}#4}}\vspace{-\baselineskip}}
    }
    

    
    % Prevent overflowing lines due to hard-to-break entities
    \sloppy
    % Setup hyperref package
    \hypersetup{
      breaklinks=true,  % so long urls are correctly broken across lines
      colorlinks=true,
      urlcolor=urlcolor,
      linkcolor=linkcolor,
      citecolor=citecolor,
      }
    % Slightly bigger margins than the latex defaults
    
    \geometry{verbose,tmargin=1in,bmargin=1in,lmargin=1in,rmargin=1in}
    
    

\begin{document}
    
    \maketitle
    
    

    
    \hypertarget{assignment-8}{%
\section{Assignment 8}\label{assignment-8}}

\hypertarget{rebecca-hawthorne}{%
\subsubsection{Rebecca Hawthorne}\label{rebecca-hawthorne}}

\hypertarget{section}{%
\subsubsection{8/23/2023}\label{section}}

    \hypertarget{the-libraries-you-will-use-are-already-loaded-for-you-below}{%
\subsubsection{The libraries you will use are already loaded for you
below}\label{the-libraries-you-will-use-are-already-loaded-for-you-below}}

    \begin{tcolorbox}[breakable, size=fbox, boxrule=1pt, pad at break*=1mm,colback=cellbackground, colframe=cellborder]
\prompt{In}{incolor}{1}{\boxspacing}
\begin{Verbatim}[commandchars=\\\{\}]
\PY{k+kn}{import} \PY{n+nn}{pandas} \PY{k}{as} \PY{n+nn}{pd}
\PY{k+kn}{import} \PY{n+nn}{numpy} \PY{k}{as} \PY{n+nn}{np}
\PY{k+kn}{import} \PY{n+nn}{seaborn} \PY{k}{as} \PY{n+nn}{sns}
\PY{k+kn}{import} \PY{n+nn}{matplotlib}\PY{n+nn}{.}\PY{n+nn}{pyplot} \PY{k}{as} \PY{n+nn}{plt}
\PY{k+kn}{from} \PY{n+nn}{itertools} \PY{k+kn}{import} \PY{n}{chain}
\end{Verbatim}
\end{tcolorbox}

    \hypertarget{question-1}{%
\subsection{Question 1}\label{question-1}}

Read in the two Netflix CSV files from /Data/Netflix as pandas
dataframes. Print the number of unique genres. This is not as simple as
it sounds. You cannot simply find the length of
\texttt{titles{[}\textquotesingle{}genres\textquotesingle{}{]}.unique()}.
You must convert the output of that code to a list, iterate over that
list and replace the following characters:
\texttt{{[}{]}\textquotesingle{},}. Once you have them replace you can
split the individual strings to list items and flatten the list. I have
already imported the \texttt{chain()} function for you to flatten the
list. Look up the documentation to see its usage. There are 19 unique
genres, but I want you to write the code to find them.

    \begin{tcolorbox}[breakable, size=fbox, boxrule=1pt, pad at break*=1mm,colback=cellbackground, colframe=cellborder]
\prompt{In}{incolor}{3}{\boxspacing}
\begin{Verbatim}[commandchars=\\\{\}]
\PY{c+c1}{\PYZsh{}read in Netflix files}
\PY{n}{Netflix\PYZus{}credit\PYZus{}df} \PY{o}{=} \PY{n}{pd}\PY{o}{.}\PY{n}{read\PYZus{}csv}\PY{p}{(}\PY{l+s+s1}{\PYZsq{}}\PY{l+s+s1}{Data/Netflix/credits.csv}\PY{l+s+s1}{\PYZsq{}}\PY{p}{)}

\PY{n}{Netflix\PYZus{}title\PYZus{}df} \PY{o}{=} \PY{n}{pd}\PY{o}{.}\PY{n}{read\PYZus{}csv}\PY{p}{(}\PY{l+s+s1}{\PYZsq{}}\PY{l+s+s1}{Data/Netflix/titles.csv}\PY{l+s+s1}{\PYZsq{}}\PY{p}{)}

\PY{c+c1}{\PYZsh{}look at data types}
\PY{n}{Netflix\PYZus{}title\PYZus{}df}\PY{o}{.}\PY{n}{dtypes}

\PY{n}{unique\PYZus{}genres\PYZus{}array}\PY{o}{=}\PY{n}{Netflix\PYZus{}title\PYZus{}df}\PY{p}{[}\PY{l+s+s1}{\PYZsq{}}\PY{l+s+s1}{genres}\PY{l+s+s1}{\PYZsq{}}\PY{p}{]}\PY{o}{.}\PY{n}{unique}\PY{p}{(}\PY{p}{)}

\PY{c+c1}{\PYZsh{}list of all characters to remove}
\PY{n}{char\PYZus{}remove}\PY{o}{=}\PY{p}{[}\PY{l+s+s1}{\PYZsq{}}\PY{l+s+s1}{[}\PY{l+s+s1}{\PYZsq{}}\PY{p}{,} \PY{l+s+s1}{\PYZsq{}}\PY{l+s+s1}{]}\PY{l+s+s1}{\PYZsq{}}\PY{p}{,} \PY{l+s+s1}{\PYZsq{}}\PY{l+s+se}{\PYZbs{}\PYZsq{}}\PY{l+s+s1}{\PYZsq{}}\PY{p}{,} \PY{l+s+s1}{\PYZsq{}}\PY{l+s+s1}{,}\PY{l+s+s1}{\PYZsq{}}\PY{p}{]}

\PY{c+c1}{\PYZsh{}go through and remove each one}
\PY{k}{for} \PY{n}{char} \PY{o+ow}{in} \PY{n}{char\PYZus{}remove}\PY{p}{:}
     \PY{n}{unique\PYZus{}genres\PYZus{}array}\PY{o}{=} \PY{p}{[}\PY{n}{ele}\PY{o}{.}\PY{n}{replace}\PY{p}{(}\PY{n}{char}\PY{p}{,} \PY{l+s+s1}{\PYZsq{}}\PY{l+s+s1}{\PYZsq{}}\PY{p}{)} \PY{k}{for} \PY{n}{ele} \PY{o+ow}{in} \PY{n}{unique\PYZus{}genres\PYZus{}array}\PY{p}{]}


\PY{c+c1}{\PYZsh{}empty list}
\PY{n}{unique\PYZus{}genres\PYZus{}list} \PY{o}{=} \PY{n+nb}{list}\PY{p}{(}\PY{p}{)}

\PY{k}{for} \PY{n}{element} \PY{o+ow}{in} \PY{n}{unique\PYZus{}genres\PYZus{}array}\PY{p}{:}
    \PY{n}{unique\PYZus{}genres\PYZus{}list}\PY{o}{.}\PY{n}{append}\PY{p}{(}\PY{n}{element}\PY{o}{.}\PY{n}{split}\PY{p}{(}\PY{l+s+s1}{\PYZsq{}}\PY{l+s+s1}{ }\PY{l+s+s1}{\PYZsq{}}\PY{p}{)}\PY{p}{)}

\PY{c+c1}{\PYZsh{}chain lists together}
\PY{n}{unique\PYZus{}genres\PYZus{}list}\PY{o}{=}\PY{n+nb}{list}\PY{p}{(}\PY{n}{chain}\PY{p}{(}\PY{o}{*}\PY{n}{unique\PYZus{}genres\PYZus{}list}\PY{p}{)}\PY{p}{)}

\PY{c+c1}{\PYZsh{}remove duplicates and empty spaces(where no genre was listed)}
\PY{n}{unique\PYZus{}genres\PYZus{}list} \PY{o}{=} \PY{n+nb}{list}\PY{p}{(}\PY{n+nb}{set}\PY{p}{(}\PY{n}{unique\PYZus{}genres\PYZus{}list}\PY{p}{)}\PY{p}{)}
\PY{n}{unique\PYZus{}genres\PYZus{}list}\PY{o}{.}\PY{n}{remove}\PY{p}{(}\PY{l+s+s2}{\PYZdq{}}\PY{l+s+s2}{\PYZdq{}}\PY{p}{)}

\PY{n+nb}{print}\PY{p}{(}\PY{n}{unique\PYZus{}genres\PYZus{}list}\PY{p}{)}
\end{Verbatim}
\end{tcolorbox}

    \begin{Verbatim}[commandchars=\\\{\}]
['scifi', 'reality', 'crime', 'fantasy', 'comedy', 'sport', 'european',
'history', 'horror', 'action', 'animation', 'drama', 'romance', 'documentation',
'music', 'western', 'thriller', 'war', 'family']
    \end{Verbatim}

    \hypertarget{question-2}{%
\subsection{Question 2}\label{question-2}}

Print the release year and the imdb score of the highest average score
of all movies by year. This is trickier than it sounds. To do this you
will need to aggregate the means by year. If you use the simple method
you will get a pandas series. The series will need to be converted to a
dataframe and the index will need to be set as a column (release year).
Once you have done that you can find the numerical index with the
highest average imdb score.

    \begin{tcolorbox}[breakable, size=fbox, boxrule=1pt, pad at break*=1mm,colback=cellbackground, colframe=cellborder]
\prompt{In}{incolor}{4}{\boxspacing}
\begin{Verbatim}[commandchars=\\\{\}]
\PY{c+c1}{\PYZsh{}find mean imdb scores, grouped by release year}
\PY{n}{mean\PYZus{}imdb\PYZus{}score\PYZus{}by\PYZus{}year} \PY{o}{=} \PY{n}{Netflix\PYZus{}title\PYZus{}df}\PY{o}{.}\PY{n}{groupby}\PY{p}{(}\PY{l+s+s1}{\PYZsq{}}\PY{l+s+s1}{release\PYZus{}year}\PY{l+s+s1}{\PYZsq{}}\PY{p}{)}\PY{p}{[}\PY{l+s+s1}{\PYZsq{}}\PY{l+s+s1}{imdb\PYZus{}score}\PY{l+s+s1}{\PYZsq{}}\PY{p}{]}\PY{o}{.}\PY{n}{mean}\PY{p}{(}\PY{p}{)}

\PY{c+c1}{\PYZsh{}convert to a dataframe}
\PY{n}{mean\PYZus{}imdb\PYZus{}score\PYZus{}by\PYZus{}year\PYZus{}df} \PY{o}{=} \PY{n}{pd}\PY{o}{.}\PY{n}{DataFrame}\PY{p}{(}\PY{n}{mean\PYZus{}imdb\PYZus{}score\PYZus{}by\PYZus{}year}\PY{p}{)}

\PY{n}{max\PYZus{}year} \PY{o}{=} \PY{n}{mean\PYZus{}imdb\PYZus{}score\PYZus{}by\PYZus{}year\PYZus{}df}\PY{o}{.}\PY{n}{idxmax}\PY{p}{(}\PY{n}{skipna}\PY{o}{=}\PY{k+kc}{True}\PY{p}{)}
\PY{n}{max\PYZus{}score} \PY{o}{=} \PY{n}{mean\PYZus{}imdb\PYZus{}score\PYZus{}by\PYZus{}year\PYZus{}df}\PY{o}{.}\PY{n}{max}\PY{p}{(}\PY{n}{skipna}\PY{o}{=}\PY{k+kc}{True}\PY{p}{)}

\PY{n+nb}{print}\PY{p}{(}\PY{l+s+s1}{\PYZsq{}}\PY{l+s+s1}{In the year }\PY{l+s+s1}{\PYZsq{}}\PY{o}{+} \PY{n+nb}{str}\PY{p}{(}\PY{n}{max\PYZus{}year}\PY{p}{[}\PY{l+s+s1}{\PYZsq{}}\PY{l+s+s1}{imdb\PYZus{}score}\PY{l+s+s1}{\PYZsq{}}\PY{p}{]}\PY{p}{)}\PY{o}{+} 
      \PY{l+s+s2}{\PYZdq{}}\PY{l+s+s2}{, the highest average IMDB score, }\PY{l+s+s2}{\PYZdq{}} 
      \PY{o}{+} \PY{n+nb}{str}\PY{p}{(}\PY{n}{max\PYZus{}score}\PY{p}{[}\PY{l+s+s1}{\PYZsq{}}\PY{l+s+s1}{imdb\PYZus{}score}\PY{l+s+s1}{\PYZsq{}}\PY{p}{]}\PY{p}{)}\PY{o}{+}\PY{l+s+s2}{\PYZdq{}}\PY{l+s+s2}{, was achieved.}\PY{l+s+s2}{\PYZdq{}}\PY{p}{)}
\end{Verbatim}
\end{tcolorbox}

    \begin{Verbatim}[commandchars=\\\{\}]
In the year 1985, the highest average IMDB score, 8.0, was achieved.
    \end{Verbatim}

    \hypertarget{question-3}{%
\subsection{Question 3}\label{question-3}}

There were 208 actors in the movie with the most credited actors. What
is the title of that movie? Nulls and NaN values do not count.

    \begin{tcolorbox}[breakable, size=fbox, boxrule=1pt, pad at break*=1mm,colback=cellbackground, colframe=cellborder]
\prompt{In}{incolor}{ }{\boxspacing}
\begin{Verbatim}[commandchars=\\\{\}]
\PY{c+c1}{\PYZsh{}get just actors (remove directors)}
\PY{n}{Netflix\PYZus{}credit\PYZus{}actors\PYZus{}df} \PY{o}{=} \PY{n}{Netflix\PYZus{}credit\PYZus{}df}\PY{p}{[}\PY{n}{Netflix\PYZus{}credit\PYZus{}df}\PY{p}{[}\PY{l+s+s2}{\PYZdq{}}\PY{l+s+s2}{role}\PY{l+s+s2}{\PYZdq{}}\PY{p}{]} \PY{o}{==}\PY{l+s+s1}{\PYZsq{}}\PY{l+s+s1}{ACTOR}\PY{l+s+s1}{\PYZsq{}}\PY{p}{]}

\PY{c+c1}{\PYZsh{}count actors by movie id}
\PY{n}{total\PYZus{}actors\PYZus{}by\PYZus{}id} \PY{o}{=} \PY{n}{Netflix\PYZus{}credit\PYZus{}actors\PYZus{}df}\PY{o}{.}\PY{n}{groupby}\PY{p}{(}\PY{p}{[}\PY{l+s+s1}{\PYZsq{}}\PY{l+s+s1}{id}\PY{l+s+s1}{\PYZsq{}}\PY{p}{]}\PY{p}{)}\PY{p}{[}\PY{l+s+s1}{\PYZsq{}}\PY{l+s+s1}{person\PYZus{}id}\PY{l+s+s1}{\PYZsq{}}\PY{p}{]}\PY{o}{.}\PY{n}{count}\PY{p}{(}\PY{p}{)}

\PY{c+c1}{\PYZsh{}join to Netflix\PYZus{}title\PYZus{}df }
\PY{n}{Netflix\PYZus{}title\PYZus{}and\PYZus{}actor} \PY{o}{=} \PY{n}{Netflix\PYZus{}title\PYZus{}df}\PY{o}{.}\PY{n}{merge}\PY{p}{(}\PY{n}{total\PYZus{}actors\PYZus{}by\PYZus{}id}\PY{p}{,} \PY{n}{how}\PY{o}{=}\PY{l+s+s1}{\PYZsq{}}\PY{l+s+s1}{outer}\PY{l+s+s1}{\PYZsq{}}\PY{p}{,} \PY{n}{on}\PY{o}{=}\PY{l+s+s1}{\PYZsq{}}\PY{l+s+s1}{id}\PY{l+s+s1}{\PYZsq{}}\PY{p}{)}

\PY{n}{largest\PYZus{}actor\PYZus{}index} \PY{o}{=} \PY{n}{Netflix\PYZus{}title\PYZus{}and\PYZus{}actor}\PY{p}{[}\PY{l+s+s1}{\PYZsq{}}\PY{l+s+s1}{person\PYZus{}id}\PY{l+s+s1}{\PYZsq{}}\PY{p}{]}\PY{o}{.}\PY{n}{idxmax}\PY{p}{(}\PY{n}{skipna}\PY{o}{=}\PY{k+kc}{True}\PY{p}{)}

\PY{n+nb}{print}\PY{p}{(}\PY{l+s+s2}{\PYZdq{}}\PY{l+s+s2}{The movie with the largest number of actors is }\PY{l+s+s2}{\PYZdq{}} 
      \PY{o}{+} \PY{n}{Netflix\PYZus{}title\PYZus{}and\PYZus{}actor}\PY{o}{.}\PY{n}{loc}\PY{p}{[}\PY{n}{largest\PYZus{}actor\PYZus{}index}\PY{p}{,} \PY{l+s+s1}{\PYZsq{}}\PY{l+s+s1}{title}\PY{l+s+s1}{\PYZsq{}}\PY{p}{]}\PY{p}{)}
\end{Verbatim}
\end{tcolorbox}

    \hypertarget{question-4}{%
\subsection{Question 4}\label{question-4}}

Which movie has the highest IMDB score for the actor Robert De Niro?
What year was it made? Create a kdeplot (kernel density estimation to
show the distribution of his IMDB movie scores.

    \begin{tcolorbox}[breakable, size=fbox, boxrule=1pt, pad at break*=1mm,colback=cellbackground, colframe=cellborder]
\prompt{In}{incolor}{5}{\boxspacing}
\begin{Verbatim}[commandchars=\\\{\}]
\PY{c+c1}{\PYZsh{}df of Robert De Niro credits}
\PY{n}{Robert\PYZus{}De\PYZus{}Niro} \PY{o}{=} \PY{n}{Netflix\PYZus{}credit\PYZus{}df}\PY{p}{[}\PY{n}{Netflix\PYZus{}credit\PYZus{}df}\PY{p}{[}\PY{l+s+s2}{\PYZdq{}}\PY{l+s+s2}{name}\PY{l+s+s2}{\PYZdq{}}\PY{p}{]} \PY{o}{==}\PY{l+s+s1}{\PYZsq{}}\PY{l+s+s1}{Robert De Niro}\PY{l+s+s1}{\PYZsq{}}\PY{p}{]}

\PY{c+c1}{\PYZsh{}merge to titles data only keeping Robert De Niro films}
\PY{n}{Robert\PYZus{}De\PYZus{}Niro\PYZus{}titles} \PY{o}{=} \PY{n}{Robert\PYZus{}De\PYZus{}Niro}\PY{o}{.}\PY{n}{merge}\PY{p}{(}\PY{n}{Netflix\PYZus{}title\PYZus{}df}\PY{p}{,} \PY{n}{how}\PY{o}{=}\PY{l+s+s1}{\PYZsq{}}\PY{l+s+s1}{left}\PY{l+s+s1}{\PYZsq{}}\PY{p}{,} \PY{n}{on}\PY{o}{=}\PY{l+s+s1}{\PYZsq{}}\PY{l+s+s1}{id}\PY{l+s+s1}{\PYZsq{}}\PY{p}{)}

\PY{c+c1}{\PYZsh{}get index of max imdb score}
\PY{n}{max\PYZus{}imdb\PYZus{}RDN\PYZus{}index} \PY{o}{=} \PY{n}{Robert\PYZus{}De\PYZus{}Niro\PYZus{}titles}\PY{p}{[}\PY{l+s+s1}{\PYZsq{}}\PY{l+s+s1}{imdb\PYZus{}score}\PY{l+s+s1}{\PYZsq{}}\PY{p}{]}\PY{o}{.}\PY{n}{idxmax}\PY{p}{(}\PY{n}{skipna}\PY{o}{=}\PY{k+kc}{True}\PY{p}{)}

\PY{n+nb}{print}\PY{p}{(}\PY{l+s+s1}{\PYZsq{}}\PY{l+s+s1}{The Robert De Niro movie with the highest IMDB core is }\PY{l+s+s1}{\PYZsq{}} 
      \PY{o}{+}\PY{n+nb}{str}\PY{p}{(} \PY{n}{Robert\PYZus{}De\PYZus{}Niro\PYZus{}titles}\PY{o}{.}\PY{n}{loc}\PY{p}{[}\PY{n}{max\PYZus{}imdb\PYZus{}RDN\PYZus{}index}\PY{p}{,} \PY{l+s+s1}{\PYZsq{}}\PY{l+s+s1}{title}\PY{l+s+s1}{\PYZsq{}}\PY{p}{]}\PY{p}{)}
      \PY{o}{+} \PY{l+s+s1}{\PYZsq{}}\PY{l+s+s1}{made in }\PY{l+s+s1}{\PYZsq{}}
      \PY{o}{+} \PY{n+nb}{str}\PY{p}{(}\PY{n}{Robert\PYZus{}De\PYZus{}Niro\PYZus{}titles}\PY{o}{.}\PY{n}{loc}\PY{p}{[}\PY{n}{max\PYZus{}imdb\PYZus{}RDN\PYZus{}index}\PY{p}{,} \PY{l+s+s1}{\PYZsq{}}\PY{l+s+s1}{release\PYZus{}year}\PY{l+s+s1}{\PYZsq{}}\PY{p}{]}\PY{p}{)}\PY{p}{)}



\PY{n}{sns}\PY{o}{.}\PY{n}{kdeplot}\PY{p}{(}\PY{n}{x}\PY{o}{=}\PY{l+s+s1}{\PYZsq{}}\PY{l+s+s1}{imdb\PYZus{}score}\PY{l+s+s1}{\PYZsq{}}
           \PY{p}{,}\PY{n}{data}\PY{o}{=}\PY{n}{Robert\PYZus{}De\PYZus{}Niro\PYZus{}titles}\PY{p}{)}

\PY{n}{plt}\PY{o}{.}\PY{n}{title}\PY{p}{(}\PY{l+s+s1}{\PYZsq{}}\PY{l+s+s1}{Density of IMDB Scores for Robert De Niro}\PY{l+s+s1}{\PYZsq{}}\PY{p}{)}
\PY{n}{plt}\PY{o}{.}\PY{n}{xlabel}\PY{p}{(}\PY{l+s+s1}{\PYZsq{}}\PY{l+s+s1}{IMDB Score}\PY{l+s+s1}{\PYZsq{}}\PY{p}{)}
\end{Verbatim}
\end{tcolorbox}

    \begin{Verbatim}[commandchars=\\\{\}]
The Robert De Niro movie with the highest IMDB core is Taxi Drivermade in 1976
    \end{Verbatim}

            \begin{tcolorbox}[breakable, size=fbox, boxrule=.5pt, pad at break*=1mm, opacityfill=0]
\prompt{Out}{outcolor}{5}{\boxspacing}
\begin{Verbatim}[commandchars=\\\{\}]
Text(0.5, 0, 'IMDB Score')
\end{Verbatim}
\end{tcolorbox}
        
    \begin{center}
    \adjustimage{max size={0.9\linewidth}{0.9\paperheight}}{output_10_2.png}
    \end{center}
    { \hspace*{\fill} \\}
    
    \hypertarget{question-5}{%
\subsection{Question 5}\label{question-5}}

Create two new boolean columns in the titles dataframe that are true
when the description contains war or gangster. Call these columns
\texttt{war\_movies} and \texttt{gangster\_movies}. How many movies are
there in both categories? Which category has a higher average IMDB
score? Show the IMDB score kernel density estimations of both
categories.

    \begin{tcolorbox}[breakable, size=fbox, boxrule=1pt, pad at break*=1mm,colback=cellbackground, colframe=cellborder]
\prompt{In}{incolor}{11}{\boxspacing}
\begin{Verbatim}[commandchars=\\\{\}]
\PY{c+c1}{\PYZsh{}create new columns for war and gangster movies}
\PY{n}{Netflix\PYZus{}title\PYZus{}df}\PY{p}{[}\PY{p}{[}\PY{l+s+s1}{\PYZsq{}}\PY{l+s+s1}{war\PYZus{}movies}\PY{l+s+s1}{\PYZsq{}}\PY{p}{,} \PY{l+s+s1}{\PYZsq{}}\PY{l+s+s1}{gangster\PYZus{}movies}\PY{l+s+s1}{\PYZsq{}}\PY{p}{]}\PY{p}{]} \PY{o}{=} \PY{l+s+s1}{\PYZsq{}}\PY{l+s+s1}{\PYZsq{}}

\PY{c+c1}{\PYZsh{}look for war and gangster in decription and set truth value of war and gangster columns}

\PY{k}{for} \PY{n}{idx} \PY{o+ow}{in} \PY{n+nb}{range}\PY{p}{(}\PY{n+nb}{len}\PY{p}{(}\PY{n}{Netflix\PYZus{}title\PYZus{}df}\PY{p}{)}\PY{p}{)}\PY{p}{:}
    \PY{n}{Netflix\PYZus{}title\PYZus{}df}\PY{p}{[}\PY{l+s+s1}{\PYZsq{}}\PY{l+s+s1}{war\PYZus{}movies}\PY{l+s+s1}{\PYZsq{}}\PY{p}{]}\PY{p}{[}\PY{n}{idx}\PY{p}{]} \PY{o}{=} \PY{l+s+s1}{\PYZsq{}}\PY{l+s+s1}{ war}\PY{l+s+s1}{\PYZsq{}} \PY{o+ow}{in} \PY{n+nb}{str}\PY{p}{(}\PY{n}{Netflix\PYZus{}title\PYZus{}df}\PY{p}{[}\PY{l+s+s1}{\PYZsq{}}\PY{l+s+s1}{description}\PY{l+s+s1}{\PYZsq{}}\PY{p}{]}\PY{p}{[}\PY{n}{idx}\PY{p}{]}\PY{p}{)}\PY{o}{.}\PY{n}{lower}\PY{p}{(}\PY{p}{)}
    \PY{n}{Netflix\PYZus{}title\PYZus{}df}\PY{p}{[}\PY{l+s+s1}{\PYZsq{}}\PY{l+s+s1}{gangster\PYZus{}movies}\PY{l+s+s1}{\PYZsq{}}\PY{p}{]}\PY{p}{[}\PY{n}{idx}\PY{p}{]} \PY{o}{=} \PY{l+s+s1}{\PYZsq{}}\PY{l+s+s1}{gangster}\PY{l+s+s1}{\PYZsq{}} \PY{o+ow}{in} \PY{n+nb}{str}\PY{p}{(}\PY{n}{Netflix\PYZus{}title\PYZus{}df}\PY{p}{[}\PY{l+s+s1}{\PYZsq{}}\PY{l+s+s1}{description}\PY{l+s+s1}{\PYZsq{}}\PY{p}{]}\PY{p}{[}\PY{n}{idx}\PY{p}{]}\PY{p}{)}\PY{o}{.}\PY{n}{lower}\PY{p}{(}\PY{p}{)}

\PY{c+c1}{\PYZsh{}set count and then count how many were war and gangster}

\PY{n}{count} \PY{o}{=} \PY{l+m+mi}{0}

\PY{k}{for} \PY{n}{idx} \PY{o+ow}{in} \PY{n+nb}{range}\PY{p}{(}\PY{n+nb}{len}\PY{p}{(}\PY{n}{Netflix\PYZus{}title\PYZus{}df}\PY{p}{)}\PY{p}{)}\PY{p}{:}
    \PY{k}{if} \PY{p}{(}\PY{n}{Netflix\PYZus{}title\PYZus{}df}\PY{p}{[}\PY{l+s+s1}{\PYZsq{}}\PY{l+s+s1}{war\PYZus{}movies}\PY{l+s+s1}{\PYZsq{}}\PY{p}{]}\PY{p}{[}\PY{n}{idx}\PY{p}{]} \PY{o}{==} \PY{k+kc}{True}\PY{p}{)} \PY{o}{\PYZam{}}  \PY{p}{(}\PY{n}{Netflix\PYZus{}title\PYZus{}df}\PY{p}{[}\PY{l+s+s1}{\PYZsq{}}\PY{l+s+s1}{gangster\PYZus{}movies}\PY{l+s+s1}{\PYZsq{}}\PY{p}{]}\PY{p}{[}\PY{n}{idx}\PY{p}{]} \PY{o}{==} \PY{k+kc}{True}\PY{p}{)}\PY{p}{:}
        \PY{n}{count} \PY{o}{+}\PY{o}{=} \PY{l+m+mi}{1}
        
\PY{n+nb}{print}\PY{p}{(}\PY{l+s+s1}{\PYZsq{}}\PY{l+s+s1}{There are }\PY{l+s+s1}{\PYZsq{}}\PY{p}{,} \PY{n}{count}\PY{p}{,} \PY{l+s+s1}{\PYZsq{}}\PY{l+s+s1}{ titles with both war and gangster.}\PY{l+s+s1}{\PYZsq{}}\PY{p}{)} 

\PY{c+c1}{\PYZsh{}find mean imdb scores for both}

\PY{n}{mean\PYZus{}war\PYZus{}movie\PYZus{}imdb\PYZus{}score} \PY{o}{=} \PY{n}{Netflix\PYZus{}title\PYZus{}df}\PY{o}{.}\PY{n}{query}\PY{p}{(}\PY{l+s+s2}{\PYZdq{}}\PY{l+s+s2}{war\PYZus{}movies == True}\PY{l+s+s2}{\PYZdq{}}\PY{p}{)}\PY{o}{.}\PY{n}{groupby}\PY{p}{(}\PY{l+s+s1}{\PYZsq{}}\PY{l+s+s1}{war\PYZus{}movies}\PY{l+s+s1}{\PYZsq{}}\PY{p}{)}\PY{p}{[}\PY{l+s+s1}{\PYZsq{}}\PY{l+s+s1}{imdb\PYZus{}score}\PY{l+s+s1}{\PYZsq{}}\PY{p}{]}\PY{o}{.}\PY{n}{mean}\PY{p}{(}\PY{p}{)}

\PY{n}{mean\PYZus{}gangster\PYZus{}movie\PYZus{}imdb\PYZus{}score} \PY{o}{=} \PY{n}{Netflix\PYZus{}title\PYZus{}df}\PY{o}{.}\PY{n}{query}\PY{p}{(}\PY{l+s+s2}{\PYZdq{}}\PY{l+s+s2}{gangster\PYZus{}movies == True}\PY{l+s+s2}{\PYZdq{}}\PY{p}{)}\PY{o}{.}\PY{n}{groupby}\PY{p}{(}\PY{l+s+s1}{\PYZsq{}}\PY{l+s+s1}{gangster\PYZus{}movies}\PY{l+s+s1}{\PYZsq{}}\PY{p}{)}\PY{p}{[}\PY{l+s+s1}{\PYZsq{}}\PY{l+s+s1}{imdb\PYZus{}score}\PY{l+s+s1}{\PYZsq{}}\PY{p}{]}\PY{o}{.}\PY{n}{mean}\PY{p}{(}\PY{p}{)}


 
\PY{n+nb}{print}\PY{p}{(}\PY{l+s+s1}{\PYZsq{}}\PY{l+s+s1}{The Average IMDB Score for war\PYZus{}movies is }\PY{l+s+s1}{\PYZsq{}}
      \PY{o}{+}\PY{n+nb}{str}\PY{p}{(}\PY{n}{mean\PYZus{}war\PYZus{}movie\PYZus{}imdb\PYZus{}score}\PY{p}{[}\PY{l+m+mi}{0}\PY{p}{]}\PY{p}{)}
      \PY{o}{+} \PY{l+s+s1}{\PYZsq{}}\PY{l+s+s1}{ and the mean imdb score for gangster\PYZus{}movies is }\PY{l+s+s1}{\PYZsq{}} 
      \PY{o}{+}\PY{n+nb}{str}\PY{p}{(}\PY{n}{mean\PYZus{}gangster\PYZus{}movie\PYZus{}imdb\PYZus{}score}\PY{p}{[}\PY{l+m+mi}{0}\PY{p}{]}\PY{p}{)}
      \PY{o}{+}\PY{l+s+s2}{\PYZdq{}}\PY{l+s+s2}{, therefore war movies have a higher average IMDB score.}\PY{l+s+s2}{\PYZdq{}}\PY{p}{)} 


\PY{c+c1}{\PYZsh{}creat kde plot}
\PY{n}{sns}\PY{o}{.}\PY{n}{kdeplot}\PY{p}{(}\PY{n}{x} \PY{o}{=} \PY{l+s+s1}{\PYZsq{}}\PY{l+s+s1}{imdb\PYZus{}score}\PY{l+s+s1}{\PYZsq{}}
           \PY{p}{,}\PY{n}{data} \PY{o}{=} \PY{n}{Netflix\PYZus{}title\PYZus{}df}\PY{o}{.}\PY{n}{query}\PY{p}{(}\PY{l+s+s1}{\PYZsq{}}\PY{l+s+s1}{war\PYZus{}movies == True}\PY{l+s+s1}{\PYZsq{}}\PY{p}{)}\PY{p}{)}
\PY{n}{sns}\PY{o}{.}\PY{n}{kdeplot}\PY{p}{(}\PY{n}{x} \PY{o}{=} \PY{l+s+s1}{\PYZsq{}}\PY{l+s+s1}{imdb\PYZus{}score}\PY{l+s+s1}{\PYZsq{}}
           \PY{p}{,}\PY{n}{data} \PY{o}{=} \PY{n}{Netflix\PYZus{}title\PYZus{}df}\PY{o}{.}\PY{n}{query}\PY{p}{(}\PY{l+s+s1}{\PYZsq{}}\PY{l+s+s1}{gangster\PYZus{}movies == True}\PY{l+s+s1}{\PYZsq{}}\PY{p}{)}\PY{p}{)}
\PY{n}{plt}\PY{o}{.}\PY{n}{title}\PY{p}{(}\PY{l+s+s1}{\PYZsq{}}\PY{l+s+s1}{Density of IMDB Scores for War and Gangster movies}\PY{l+s+s1}{\PYZsq{}}\PY{p}{)}
\PY{n}{plt}\PY{o}{.}\PY{n}{xlabel}\PY{p}{(}\PY{l+s+s1}{\PYZsq{}}\PY{l+s+s1}{IMDB Score}\PY{l+s+s1}{\PYZsq{}}\PY{p}{)}

\PY{n}{plt}\PY{o}{.}\PY{n}{legend}\PY{p}{(}\PY{n}{title}\PY{o}{=}\PY{l+s+s1}{\PYZsq{}}\PY{l+s+s1}{Movie Type}\PY{l+s+s1}{\PYZsq{}}\PY{p}{,} \PY{n}{loc}\PY{o}{=}\PY{l+s+s1}{\PYZsq{}}\PY{l+s+s1}{upper left}\PY{l+s+s1}{\PYZsq{}}\PY{p}{,} 
           \PY{n}{labels} \PY{o}{=} \PY{p}{[}\PY{l+s+s1}{\PYZsq{}}\PY{l+s+s1}{War Movies}\PY{l+s+s1}{\PYZsq{}}\PY{p}{,} \PY{l+s+s1}{\PYZsq{}}\PY{l+s+s1}{Gangster Movies}\PY{l+s+s1}{\PYZsq{}}\PY{p}{]}\PY{p}{)}  

\PY{n}{plt}\PY{o}{.}\PY{n}{show}\PY{p}{(}\PY{p}{)}
\end{Verbatim}
\end{tcolorbox}

    \begin{Verbatim}[commandchars=\\\{\}]
C:\textbackslash{}Users\textbackslash{}hawthorner\textbackslash{}AppData\textbackslash{}Local\textbackslash{}Temp\textbackslash{}ipykernel\_2604\textbackslash{}1574626028.py:7:
SettingWithCopyWarning:
A value is trying to be set on a copy of a slice from a DataFrame

See the caveats in the documentation: https://pandas.pydata.org/pandas-
docs/stable/user\_guide/indexing.html\#returning-a-view-versus-a-copy
  Netflix\_title\_df['war\_movies'][idx] = ' war' in
str(Netflix\_title\_df['description'][idx]).lower()
C:\textbackslash{}Users\textbackslash{}hawthorner\textbackslash{}AppData\textbackslash{}Local\textbackslash{}Temp\textbackslash{}ipykernel\_2604\textbackslash{}1574626028.py:8:
SettingWithCopyWarning:
A value is trying to be set on a copy of a slice from a DataFrame

See the caveats in the documentation: https://pandas.pydata.org/pandas-
docs/stable/user\_guide/indexing.html\#returning-a-view-versus-a-copy
  Netflix\_title\_df['gangster\_movies'][idx] = 'gangster' in
str(Netflix\_title\_df['description'][idx]).lower()
    \end{Verbatim}

    \begin{Verbatim}[commandchars=\\\{\}]
There are  3  titles with both war and gangster.
The Average IMDB Score for war\_movies is 6.863934426229508 and the mean imdb
score for gangster\_movies is 6.3914285714285715, therefore war movies have a
higher average IMDB score.
    \end{Verbatim}

    \begin{center}
    \adjustimage{max size={0.9\linewidth}{0.9\paperheight}}{output_12_2.png}
    \end{center}
    { \hspace*{\fill} \\}
    

    % Add a bibliography block to the postdoc
    
    
    
\end{document}
